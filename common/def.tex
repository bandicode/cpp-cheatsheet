
\usepackage{array}
\usepackage{makecell}

\usepackage{enumitem}

\usepackage{colortbl} % provides arrayrulecolor
\usepackage{graphicx}
\usepackage{relsize}      % provide relative font size changes
\usepackage{xspace}


\renewcommand{\baselinestretch}{.8}
\pagestyle{empty}

\makeatletter 

\makeatother

\setlength{\parindent}{0pt}

\newenvironment{tightitemize}
{ \vspace{-\topsep}
  \vspace{4pt}
  \begin{itemize}[leftmargin=0pt, itemindent=12pt]
    \setlength{\itemsep}{4pt}
    \setlength{\parskip}{0pt}
    \setlength{\parsep}{0pt}      }
{ \end{itemize} \vspace{-\topsep} \vspace{4pt} } 

%%--------------------------------------------------
%% Sections & titles
\setlength{\parskip}{0.4\baselineskip}
\newcommand{\headline}[1]{\vspace{0.2\baselineskip}\textbf{#1}}

%%--------------------------------------------------
%% Horizontal gray rules

\newenvironment{userendnotes}[1]
  {
    \vspace{1ex}
    \setlength{\parskip}{1.2\baselineskip}
  }
  { 
    \setlength{\parskip}{0.4\baselineskip}
  }

\newcommand{\hgrayrule}{\par\noindent\textcolor{gray}{\rule{0.31\textwidth}{0.4pt}}}

%%--------------------------------------------------
%% array
\renewcommand\cellgape{\Gape[4pt]}
\newcolumntype{L}[1]{>{\raggedright\let\newline\\\arraybackslash\hspace{0pt}}m{#1}}
\newcolumntype{C}[1]{>{\centering\let\newline\\\arraybackslash\hspace{0pt}}m{#1}}
\newcolumntype{R}[1]{>{\raggedleft\let\newline\\\arraybackslash\hspace{0pt}}m{#1}}

%%--------------------------------------------------
%% Title
\newcommand{\CheatsheetTitle}[1]{
  \vbox { 
    \kern -6ex 
    \mbox{ 
      \includegraphics[scale=0.15]{common/cpp_logo} 
      \hspace{1em} 
      \raisebox{3ex}{
        \vbox{ 
          \hbox{\Huge \textsc{\Cpp Cheatsheet - #1} } 
          \kern 1ex
          \hbox{ \kern 0.1em Bibliography: \href{http://isocpp.org}{isocpp.org}, \href{http://cppreference.com}{cppreference.com}, \href{https://isocpp.github.io/CppCoreGuidelines/CppCoreGuidelines}{CppCoreGuidelines}}
        }
      }
    }
  }
}

\usepackage{titlesec}
\titlespacing\section{0pt}{\parskip}{-0.5\parskip}
\titlespacing\subsection{0pt}{\parskip}{-0.5\parskip}
\titlespacing\subsubsection{0pt}{\parskip}{-0.5\parskip}


%%--------------------------------------------------
%% fix interaction between hyperref and other
%% commands
\pdfstringdefDisableCommands{\def\smaller#1{#1}}
\pdfstringdefDisableCommands{\def\textbf#1{#1}}
\pdfstringdefDisableCommands{\def\raisebox#1{}}
\pdfstringdefDisableCommands{\def\hspace#1{}}


%%--------------------------------------------------
%% URL

\usepackage{hyperref}
\hypersetup{
    colorlinks=true,
    linkcolor=blue,
    filecolor=blue,      
    urlcolor=blue,
}
 
\urlstyle{same}

\newcommand{\hrefbf}[2]{\href{#1}{\textbf{#2}}}
\newcommand{\hrefcode}[2]{\href{#1}{\tcode{#2}}}
\newcommand{\hrefrequirements}[2]{\href{#1}{\textit{#2}}}
\newcommand{\hlink}[2]{\hyperlink{#1}{#2}}
\newcommand{\hlinkb}[2]{\hyperlink{#1}{\textbf{#2}}}



%%--------------------------------------------------
%% C++

\newcommand{\Rplus}{\protect\hspace{-.1em}\protect\raisebox{.35ex}{\smaller{\smaller\textbf{+}}}}
%\newcommand{\Rplus}{\protect\hspace{-.1em}\protect\raisebox{.25ex}{+}}
%\newcommand{\Rplus}{+}
\newcommand{\Cpp}{\mbox{C\Rplus\Rplus}\xspace}

%%--------------------------------------------------
%% Info

\definecolor{darkgreen}{RGB}{0,128,0}
\definecolor{darkorange}{RGB}{224, 96, 0}

\newcommand{\tmark}[1]{{\smaller\smaller \textcolor{darkgreen}{(#1)}}}
\newcommand{\tsincecxx}[1]{\tmark{since \Cpp\/#1}}
\newcommand{\tuntilcxx}[1]{\tmark{until \Cpp\/#1}}

\newcommand{\tmarkts}[1]{{\smaller\smaller \textcolor{darkorange}{(#1)}}}


% Inline code

\newcommand{\CodeStyle}{\ttfamily}
\newcommand{\CodeStylex}[1]{\texttt{#1}}
%\newcommand{\CodeStylex}[1]{\ttfamily{#1}}
%\newcommand{\CodeStylex}[1]{{\fontfamily{normalfont}\selectfont #1}}
\newcommand{\tcode}[1]{\CodeStylex{#1}}

% Inline code (framed)

\usepackage{xcolor}
\usepackage[most]{tcolorbox}

\definecolor{boxtfcode_border}{RGB}{214,214,214}
\definecolor{boxtfcode_bg}{RGB}{245,245,245}

\newtcbox{\boxtfcode}[1][boxtfcode_border]{on line,
	%size = tight,
    top=2pt,left=1pt,right=1pt,bottom=2pt,
    colback=boxtfcode_bg, colframe=boxtfcode_border, 
	boxsep=0pt, 
	boxrule=1pt, 
	arc=2pt
	}

\newcommand{\tfcode}[1]{\mbox{\boxtfcode{\tcode{#1}}}\xspace}

% Syntax

\newcommand{\tsyntax}[1]{\textcolor{gray}{\textit{#1}}}


%%--------------------------------------------------
%% Code listings

\usepackage{listings}
\tcbuselibrary{listings,skins}

\definecolor{main-color}{rgb}{0.6627, 0.7176, 0.7764}
\definecolor{back-color}{rgb}{0.1686, 0.1686, 0.1686}
%\definecolor{string-color}{rgb}{0.3333, 0.5254, 0.345}
\definecolor{string-color}{RGB}{51, 153, 0}
%\definecolor{key-color}{rgb}{0.8, 0.47, 0.196}
\definecolor{key-color}{RGB}{0,0,221}
\definecolor{preprocessor-color}{RGB}{51, 153, 0}
\definecolor{number-color}{RGB}{0, 0, 128}
\definecolor{comment-color}{RGB}{144,144,144}

\newcommand*{\FormatDigit}[1]{\textcolor{number-color}{#1}}

\lstset{language=C++,
        basicstyle=\ttfamily,
        keywordstyle=\color{key-color}\ttfamily,
        stringstyle=\color{string-color}\ttfamily,
        xleftmargin=0.5em,
        showstringspaces=false,
        commentstyle=\itshape\ttfamily,
        columns=flexible,
        keepspaces=true,
        texcl=true,
		tabsize=4,
		aboveskip=1pt,
		belowskip=0pt,
		literate={\ \ }{{\ }}1
		{0}{{\FormatDigit{0}}}{1}%
        {1}{{\FormatDigit{1}}}{1}%
        {2}{{\FormatDigit{2}}}{1}%
        {3}{{\FormatDigit{3}}}{1}%
        {4}{{\FormatDigit{4}}}{1}%
        {5}{{\FormatDigit{5}}}{1}%
        {6}{{\FormatDigit{6}}}{1}%
        {7}{{\FormatDigit{7}}}{1}%
        {8}{{\FormatDigit{8}}}{1}%
        {9}{{\FormatDigit{9}}}{1}%
        {.0}{{\FormatDigit{.0}}}{2}% Following is to ensure that only periods
        {.1}{{\FormatDigit{.1}}}{2}% followed by a digit are changed.
        {.2}{{\FormatDigit{.2}}}{2}%
        {.3}{{\FormatDigit{.3}}}{2}%
        {.4}{{\FormatDigit{.4}}}{2}%
        {.5}{{\FormatDigit{.5}}}{2}%
        {.6}{{\FormatDigit{.6}}}{2}%
        {.7}{{\FormatDigit{.7}}}{2}%
        {.8}{{\FormatDigit{.8}}}{2}%
        {.9}{{\FormatDigit{.9}}}{2}%
        %{,}{{\FormatDigit{,}}{1}% depends if you want the "," in color
		}

\newcommand{\CodeBlockSetup}{
 \lstset{escapechar=@}
}

%\lstnewenvironment{codeblock}{
%    \CodeBlockSetup
%}{}

\lstnewenvironment{hsynopsis}{
    \CodeBlockSetup
}{}

\newtcblisting{codeblock}[0]{
    top=0pt,left=2pt,right=2pt,bottom=0pt,
    colback=boxtfcode_bg, colframe=boxtfcode_border, 
	boxsep=0pt, 
	boxrule=1pt, 
	arc=2pt,
    listing only,
	listing options={language=C++,
        basicstyle=\small\CodeStyle,
        keywordstyle=\color{key-color}\small\CodeStyle,
		numberstyle=\color{number-color}\small\CodeStyle,
        stringstyle=\color{string-color}\small\CodeStyle,
        xleftmargin=0em,
        showstringspaces=false,
        %commentstyle=\itshape\rmfamily,
		commentstyle=\color{comment-color}\small\CodeStyle,
		directivestyle=\color{preprocessor-color}\small\CodeStyle,
        columns=flexible,
        keepspaces=true,
        texcl=true,
		tabsize=4,
        literate={\ \ }{{\ }}1
		{0}{{\FormatDigit{0}}}{1}%
        {1}{{\FormatDigit{1}}}{1}%
        {2}{{\FormatDigit{2}}}{1}%
        {3}{{\FormatDigit{3}}}{1}%
        {4}{{\FormatDigit{4}}}{1}%
        {5}{{\FormatDigit{5}}}{1}%
        {6}{{\FormatDigit{6}}}{1}%
        {7}{{\FormatDigit{7}}}{1}%
        {8}{{\FormatDigit{8}}}{1}%
        {9}{{\FormatDigit{9}}}{1}%
        {.0}{{\FormatDigit{.0}}}{2}% Following is to ensure that only periods
        {.1}{{\FormatDigit{.1}}}{2}% followed by a digit are changed.
        {.2}{{\FormatDigit{.2}}}{2}%
        {.3}{{\FormatDigit{.3}}}{2}%
        {.4}{{\FormatDigit{.4}}}{2}%
        {.5}{{\FormatDigit{.5}}}{2}%
        {.6}{{\FormatDigit{.6}}}{2}%
        {.7}{{\FormatDigit{.7}}}{2}%
        {.8}{{\FormatDigit{.8}}}{2}%
        {.9}{{\FormatDigit{.9}}}{2}%
        %{,}{{\FormatDigit{,}}{1}% depends if you want the "," in color
		},
}

% A code block in which single-quotes are digit separators
% rather than character literals.
\lstnewenvironment{codeblockdigitsep}{
 \CodeBlockSetup
 \lstset{deletestring=[b]{'}}
}{}

% Permit use of '@' inside codeblock blocks
\makeatletter
\newcommand{\atsign}{@}
\makeatother



