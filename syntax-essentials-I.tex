\documentclass[10pt,landscape,a4paper]{article}
\usepackage[utf8]{inputenc}
\usepackage[english]{babel}

\usepackage[T1]{fontenc}
\usepackage{lmodern}
\usepackage{tikz}
\usetikzlibrary{shapes,positioning,arrows,fit,calc,graphs,graphs.standard}
%\usepackage[nosf]{kpfonts}
%\usepackage[t1]{sourcesanspro}
\usepackage{multicol}
\usepackage{wrapfig}
\usepackage[top=10mm,bottom=10mm,left=10mm,right=10mm]{geometry}
\usepackage[framemethod=tikz]{mdframed}
%\usepackage{microtype}
\usepackage{pdfpages}

\usepackage[pdftex,
            pdfauthor={bandicode},
            pdftitle={C++ Cheatsheet - Syntax Essentials (I)},
            pdfsubject={Syntax Essentials},
            pdfkeywords={cpp,syntax},
            pdfproducer={Latex},
            pdfcreator={pdflatex}]{hyperref}


\usepackage{array}
\usepackage{makecell}

\usepackage{enumitem}

\usepackage{colortbl} % provides arrayrulecolor
\usepackage{graphicx}
\usepackage{relsize}      % provide relative font size changes
\usepackage{xspace}


\renewcommand{\baselinestretch}{.8}
\pagestyle{empty}

\makeatletter 

\makeatother

\setlength{\parindent}{0pt}

\newenvironment{tightitemize}
{ \vspace{-\topsep}
  \vspace{4pt}
  \begin{itemize}[leftmargin=0pt, itemindent=12pt]
    \setlength{\itemsep}{4pt}
    \setlength{\parskip}{0pt}
    \setlength{\parsep}{0pt}      }
{ \end{itemize} \vspace{-\topsep} \vspace{4pt} } 

%%--------------------------------------------------
%% Sections & titles
\setlength{\parskip}{0.4\baselineskip}
\newcommand{\headline}[1]{\vspace{0.2\baselineskip}\textbf{#1}}

%%--------------------------------------------------
%% Horizontal gray rules

\newenvironment{userendnotes}[1]
  {
    \vspace{1ex}
    \setlength{\parskip}{1.2\baselineskip}
  }
  { 
    \setlength{\parskip}{0.4\baselineskip}
  }

\newcommand{\hgrayrule}{\par\noindent\textcolor{gray}{\rule{0.31\textwidth}{0.4pt}}}

%%--------------------------------------------------
%% array
\renewcommand\cellgape{\Gape[4pt]}
\newcolumntype{L}[1]{>{\raggedright\let\newline\\\arraybackslash\hspace{0pt}}m{#1}}
\newcolumntype{C}[1]{>{\centering\let\newline\\\arraybackslash\hspace{0pt}}m{#1}}
\newcolumntype{R}[1]{>{\raggedleft\let\newline\\\arraybackslash\hspace{0pt}}m{#1}}

%%--------------------------------------------------
%% Title
\newcommand{\CheatsheetTitle}[1]{
  \vbox { 
    \kern -6ex 
    \mbox{ 
      \includegraphics[scale=0.15]{common/cpp_logo} 
      \hspace{1em} 
      \raisebox{3ex}{
        \vbox{ 
          \hbox{\Huge \textsc{\Cpp Cheatsheet - #1} } 
          \kern 1ex
          \hbox{ \kern 0.1em Bibliography: \href{http://isocpp.org}{isocpp.org}, \href{http://cppreference.com}{cppreference.com}, \href{https://isocpp.github.io/CppCoreGuidelines/CppCoreGuidelines}{CppCoreGuidelines}}
        }
      }
    }
  }
}

\usepackage{titlesec}
\titlespacing\section{0pt}{\parskip}{-0.5\parskip}
\titlespacing\subsection{0pt}{\parskip}{-0.5\parskip}
\titlespacing\subsubsection{0pt}{\parskip}{-0.5\parskip}


%%--------------------------------------------------
%% fix interaction between hyperref and other
%% commands
\pdfstringdefDisableCommands{\def\smaller#1{#1}}
\pdfstringdefDisableCommands{\def\textbf#1{#1}}
\pdfstringdefDisableCommands{\def\raisebox#1{}}
\pdfstringdefDisableCommands{\def\hspace#1{}}


%%--------------------------------------------------
%% URL

\usepackage{hyperref}
\hypersetup{
    colorlinks=true,
    linkcolor=blue,
    filecolor=blue,      
    urlcolor=blue,
}
 
\urlstyle{same}

\newcommand{\hrefbf}[2]{\href{#1}{\textbf{#2}}}
\newcommand{\hrefcode}[2]{\href{#1}{\tcode{#2}}}
\newcommand{\hrefrequirements}[2]{\href{#1}{\textit{#2}}}
\newcommand{\hlink}[2]{\hyperlink{#1}{#2}}
\newcommand{\hlinkb}[2]{\hyperlink{#1}{\textbf{#2}}}



%%--------------------------------------------------
%% C++

\newcommand{\Rplus}{\protect\hspace{-.1em}\protect\raisebox{.35ex}{\smaller{\smaller\textbf{+}}}}
%\newcommand{\Rplus}{\protect\hspace{-.1em}\protect\raisebox{.25ex}{+}}
%\newcommand{\Rplus}{+}
\newcommand{\Cpp}{\mbox{C\Rplus\Rplus}\xspace}

%%--------------------------------------------------
%% Info

\definecolor{darkgreen}{RGB}{0,128,0}
\definecolor{darkorange}{RGB}{224, 96, 0}

\newcommand{\tmark}[1]{{\smaller\smaller \textcolor{darkgreen}{(#1)}}}
\newcommand{\tsincecxx}[1]{\tmark{since \Cpp\/#1}}
\newcommand{\tuntilcxx}[1]{\tmark{until \Cpp\/#1}}

\newcommand{\tmarkts}[1]{{\smaller\smaller \textcolor{darkorange}{(#1)}}}


% Inline code

\newcommand{\CodeStyle}{\ttfamily}
\newcommand{\CodeStylex}[1]{\texttt{#1}}
%\newcommand{\CodeStylex}[1]{\ttfamily{#1}}
%\newcommand{\CodeStylex}[1]{{\fontfamily{normalfont}\selectfont #1}}
\newcommand{\tcode}[1]{\CodeStylex{#1}}

% Inline code (framed)

\usepackage{xcolor}
\usepackage[most]{tcolorbox}

\definecolor{boxtfcode_border}{RGB}{214,214,214}
\definecolor{boxtfcode_bg}{RGB}{245,245,245}

\newtcbox{\boxtfcode}[1][boxtfcode_border]{on line,
	%size = tight,
    top=2pt,left=1pt,right=1pt,bottom=2pt,
    colback=boxtfcode_bg, colframe=boxtfcode_border, 
	boxsep=0pt, 
	boxrule=1pt, 
	arc=2pt
	}

\newcommand{\tfcode}[1]{\mbox{\boxtfcode{\tcode{#1}}}\xspace}

% Syntax

\newcommand{\tsyntax}[1]{\textcolor{gray}{\textit{#1}}}


%%--------------------------------------------------
%% Code listings

\usepackage{listings}
\tcbuselibrary{listings,skins}

\definecolor{main-color}{rgb}{0.6627, 0.7176, 0.7764}
\definecolor{back-color}{rgb}{0.1686, 0.1686, 0.1686}
%\definecolor{string-color}{rgb}{0.3333, 0.5254, 0.345}
\definecolor{string-color}{RGB}{51, 153, 0}
%\definecolor{key-color}{rgb}{0.8, 0.47, 0.196}
\definecolor{key-color}{RGB}{0,0,221}
\definecolor{preprocessor-color}{RGB}{51, 153, 0}
\definecolor{number-color}{RGB}{0, 0, 128}
\definecolor{comment-color}{RGB}{144,144,144}

\newcommand*{\FormatDigit}[1]{\textcolor{number-color}{#1}}

\lstset{language=C++,
        basicstyle=\ttfamily,
        keywordstyle=\color{key-color}\ttfamily,
        stringstyle=\color{string-color}\ttfamily,
        xleftmargin=0.5em,
        showstringspaces=false,
        commentstyle=\itshape\ttfamily,
        columns=flexible,
        keepspaces=true,
        texcl=true,
		tabsize=4,
		aboveskip=1pt,
		belowskip=0pt,
		literate={\ \ }{{\ }}1
		{0}{{\FormatDigit{0}}}{1}%
        {1}{{\FormatDigit{1}}}{1}%
        {2}{{\FormatDigit{2}}}{1}%
        {3}{{\FormatDigit{3}}}{1}%
        {4}{{\FormatDigit{4}}}{1}%
        {5}{{\FormatDigit{5}}}{1}%
        {6}{{\FormatDigit{6}}}{1}%
        {7}{{\FormatDigit{7}}}{1}%
        {8}{{\FormatDigit{8}}}{1}%
        {9}{{\FormatDigit{9}}}{1}%
        {.0}{{\FormatDigit{.0}}}{2}% Following is to ensure that only periods
        {.1}{{\FormatDigit{.1}}}{2}% followed by a digit are changed.
        {.2}{{\FormatDigit{.2}}}{2}%
        {.3}{{\FormatDigit{.3}}}{2}%
        {.4}{{\FormatDigit{.4}}}{2}%
        {.5}{{\FormatDigit{.5}}}{2}%
        {.6}{{\FormatDigit{.6}}}{2}%
        {.7}{{\FormatDigit{.7}}}{2}%
        {.8}{{\FormatDigit{.8}}}{2}%
        {.9}{{\FormatDigit{.9}}}{2}%
        %{,}{{\FormatDigit{,}}{1}% depends if you want the "," in color
		}

\newcommand{\CodeBlockSetup}{
 \lstset{escapechar=@}
}

%\lstnewenvironment{codeblock}{
%    \CodeBlockSetup
%}{}

\lstnewenvironment{hsynopsis}{
    \CodeBlockSetup
}{}

\newtcblisting{codeblock}[0]{
    top=0pt,left=2pt,right=2pt,bottom=0pt,
    colback=boxtfcode_bg, colframe=boxtfcode_border, 
	boxsep=0pt, 
	boxrule=1pt, 
	arc=2pt,
    listing only,
	listing options={language=C++,
        basicstyle=\small\CodeStyle,
        keywordstyle=\color{key-color}\small\CodeStyle,
		numberstyle=\color{number-color}\small\CodeStyle,
        stringstyle=\color{string-color}\small\CodeStyle,
        xleftmargin=0em,
        showstringspaces=false,
        %commentstyle=\itshape\rmfamily,
		commentstyle=\color{comment-color}\small\CodeStyle,
		directivestyle=\color{preprocessor-color}\small\CodeStyle,
        columns=flexible,
        keepspaces=true,
        texcl=true,
		tabsize=4,
        literate={\ \ }{{\ }}1
		{0}{{\FormatDigit{0}}}{1}%
        {1}{{\FormatDigit{1}}}{1}%
        {2}{{\FormatDigit{2}}}{1}%
        {3}{{\FormatDigit{3}}}{1}%
        {4}{{\FormatDigit{4}}}{1}%
        {5}{{\FormatDigit{5}}}{1}%
        {6}{{\FormatDigit{6}}}{1}%
        {7}{{\FormatDigit{7}}}{1}%
        {8}{{\FormatDigit{8}}}{1}%
        {9}{{\FormatDigit{9}}}{1}%
        {.0}{{\FormatDigit{.0}}}{2}% Following is to ensure that only periods
        {.1}{{\FormatDigit{.1}}}{2}% followed by a digit are changed.
        {.2}{{\FormatDigit{.2}}}{2}%
        {.3}{{\FormatDigit{.3}}}{2}%
        {.4}{{\FormatDigit{.4}}}{2}%
        {.5}{{\FormatDigit{.5}}}{2}%
        {.6}{{\FormatDigit{.6}}}{2}%
        {.7}{{\FormatDigit{.7}}}{2}%
        {.8}{{\FormatDigit{.8}}}{2}%
        {.9}{{\FormatDigit{.9}}}{2}%
        %{,}{{\FormatDigit{,}}{1}% depends if you want the "," in color
		},
}

% A code block in which single-quotes are digit separators
% rather than character literals.
\lstnewenvironment{codeblockdigitsep}{
 \CodeBlockSetup
 \lstset{deletestring=[b]{'}}
}{}

% Permit use of '@' inside codeblock blocks
\makeatletter
\newcommand{\atsign}{@}
\makeatother





\begin{document}

\CheatsheetTitle{Syntax Essentials (I)}

\small
\begin{multicols*}{3}

\section*{Declaring \& Initializing variables}

In \Cpp, a large majority of variables are declared and initialized in one 
of the following ways.

\bgroup
\def\arraystretch{1.5}
\begin{tabular}{ l c } 
  \arrayrulecolor{gray}\hline
    \tsyntax{type} \tsyntax{name}; & (1) \\ 
  \hline
    \tsyntax{type} \tsyntax{name}(\tsyntax{expression-list}); & (2) \\ 
  \hline
    \tsyntax{type} \tsyntax{name} = \tsyntax{expression}; & (3) \\ 
  \hline
    \tsyntax{type} \tsyntax{name}\{\tsyntax{initializer-list}\}; \tsincecxx{11} & (4) \\ 
\end{tabular}
\egroup

\headline{Pointers \& References}

Let \tcode{T} be a fundamental type, a class or an enumeration name in the following paragraphs. 

We write \tcode{T*} the type \emph{pointer to T} that 
can be used to store the address in memory of a variable of type \tcode{T}.  
Pointers can have the special value \tcode{nullptr}, indicating that they do not 
point to any existing object.
If \tcode{t} is a variable of type \tcode{T*} that is not \tcode{nullptr}, 
then the expression \tcode{*t} can be used to access the object pointed to.

We write \tcode{T\&} the type \emph{reference to T}, that is an alias to an already existing value.
Variables of type \tcode{T\&} are used just like those of type \tcode{T}.
Variables of reference type must be initialized.

\tcode{T\&\&} is a special kind of reference, introduced in \Cpp11 and called a \emph{rvalue-reference}, 
that is used to extend the lifetime of temporaries and transfer resource ownership.

\headline{Variable initialization}

In \tcode{(1)}, \href{https://en.cppreference.com/w/cpp/language/default_initialization}{default initialization} applies: 
primitive types are left uninitialized and object types are default constructed. 
For all other constructs, various forms of initialization take place depending on the context. 
The most common one are presented below.

%%\headline{Value initialization}

$ \bullet $ When form \tcode{(4)} is used with an empty initializer-list, 
\href{https://en.cppreference.com/w/cpp/language/value_initialization}{value initialization} 
is performed: variables of primitive-type are zero-initialized, variables of object-type are default constructed.

%%\headline{Copy initialization}

$ \bullet $ In form \tcode{(3)}, \href{https://en.cppreference.com/w/cpp/language/copy_initialization}{copy initialization} 
looks for converting constructors or user-defined conversion to initialize the object. 

%%\headline{Direct initialization}

$ \bullet $ Form \tcode{(2)} uses an explicit set of constructor arguments to initialize the object. 
The expression-list may not be empty (as it would declare a function taking no parameter).
Unlike copy-initialization, \tcode{explicit} constructors can be used in 
\href{https://en.cppreference.com/w/cpp/language/direct_initialization}{direct initialization}.

%%\headline{List initialization}

$ \bullet $ In form \tcode{(4)},
\href{https://en.cppreference.com/w/cpp/language/list_initialization}{list initialization} 
is performed. All constructors are considered and the best is used to initialize the object. 
Constructors taking a specialization of \hrefcode{https://en.cppreference.com/w/cpp/utility/initializer_list}{std::initializer\_list} 
are preferred. 

%%\headline{Reference initialization}

$ \bullet $ Forms \tcode{(2)}, \tcode{(3)} or \tcode{(4)} may be used to initialize variable of reference type.
Since references must be initialized, form \tcode{(1)} cannot apply.

%%\headline{Aggregate initialization}

$ \bullet $ \href{https://en.cppreference.com/w/cpp/language/aggregate_initialization}{Aggregate initialization} can be used 
to initialize aggregate.

\headline{auto type deduction}

You can use \tfcode{auto} to let the compiler infer the type of your variable.
This can be really useful in template code where the actual type of an expression 
might not be known, or more generally if you do not want to bind a variable to a 
particular type (e.g. when using iterators).

\begin{codeblock}
std::vector<int> digits = {0, 1, 2, 3, 4};
auto it = digits.begin();
\end{codeblock}

\headline{const qualification}

The \tfcode{const} keyword may be used to make a variable immutable once it has 
been initialized.

\headline{Uniform initialization}

Brace-initialization can also be used in a variety of contexts (including in return 
statements and in function arguments) to initialize an object without specifying 
a type.

\begin{codeblock}
f({1, 2, 3});
return {};
\end{codeblock}

\section*{Functions \& Control Flow}

The main syntax for defining a function is the following:

\bgroup
\def\arraystretch{1.5}
\begin{tabular}{ l } 
    \tsyntax{return-type} \tsyntax{name}(\tsyntax{parameter-list}) \{ \tsyntax{statements} \} \\ 
\end{tabular}
\egroup

\subsection*{Statements}

$ \bullet $ A variable declaration is a statement.

$ \bullet $ An expression followed by a semicolon is an \tsyntax{expression-statement}.

$ \bullet $ A pair of curly braces containing zero or more statements is a \tsyntax{compound-statement}.

\subsection*{Control flow statements}

\headline{Returning from a function}

Functions having a return type that is not \tcode{void} should 
\tcode{return} a value. 
A \tsyntax{return-statement} may also be used without a value in function 
returning \tcode{void} to return earlier.

\bgroup
\def\arraystretch{1.5}
\begin{tabular}{ l } 
    \arrayrulecolor{gray}\hline
    \tcode{return;} \\ 
    \hline
    \tcode{return} \tsyntax{expression}\tcode{;}\\ 
\end{tabular}
\egroup

\begin{codeblock}
int foo(int a, int b, int c)
{
  return a + b * c;
}
\end{codeblock}

\headline{Selection statements}

Branching is usually done with \tsyntax{if-statements}, when the condition is 
of integral type, a \tsyntax{switch-statement} may also be used.

\tsyntax{if-statement}:

\bgroup
\def\arraystretch{1.5}
\begin{tabular}{ l } 
    \arrayrulecolor{gray}\hline
    \tcode{if (}\tsyntax{condition}\tcode{)} \tsyntax{statement} \\ 
    \hline
    \tcode{if (}\tsyntax{condition}\tcode{)} \tsyntax{statement} \tcode{else} \tsyntax{statement} \\ 
\end{tabular}
\egroup

\tsyntax{switch-statement}:

\bgroup
\def\arraystretch{1.5}
\begin{tabular}{ l } 
    \arrayrulecolor{gray}\hline
    \tcode{switch (}\tsyntax{condition}\tcode{)} \tsyntax{statement} \\ 
\end{tabular}
\egroup

In case of a switch, the statement is usually a compound statement that contains 
\tcode{case} and \tcode{default}; and \tfcode{break} has a special meaning.

\bgroup
\def\arraystretch{1.5}
\begin{tabular}{ l } 
    \arrayrulecolor{gray}\hline
    \tcode{case} \tsyntax{constant-expression}\tcode{:} \tsyntax{statement} \\ 
    \hline
    \tcode{default:} \tsyntax{statement} \\ 
\end{tabular}
\egroup

\headline{Iteration statements (loops)}

\bgroup
\def\arraystretch{1.5}
\begin{tabular}{ l } 
    \arrayrulecolor{gray}\hline
    \tcode{while (}\tsyntax{condition}\tcode{)} \tsyntax{statement} \\ 
    \hline
    \tcode{for (}\tsyntax{init}\tcode{; }\tsyntax{condition}\tcode{; }\tsyntax{loop}\tcode{)} \tsyntax{statement} \\ 
    \hline
    \tcode{for (}\tsyntax{type} \tsyntax{name} \tcode{:} \tsyntax{container}\tcode{)} \tsyntax{statement} \tsincecxx{11} \\ 
    \hline
    \tcode{do} \tsyntax{statement} \tcode{while (}\tsyntax{condition}\tcode{);} \\ 
\end{tabular}
\egroup

Inside loops, \tcode{continue} and \tcode{break} can be used respectively to 
stop the current iteration of the loop and start the next one, and to exit the loop.

 \bgroup
\def\arraystretch{1.5}
\begin{tabular}{ l } 
    \arrayrulecolor{gray}\hline
    \tcode{break;} \\ 
    \hline
    \tcode{continue;} \\ 
\end{tabular}
\egroup

\subsection*{Function parameters}

\headline{Passing by value / Passing by reference}

When the parameter of a function is a reference, the arguments are not copied. 
If the reference is not \tcode{const}, modifications made to the object inside 
the function are applied to the original object and therefore visible to the caller.

When the parameter is not a reference, arguments are passed by value (i.e. copied).
Modifications made to the object modifies the copy and are therefore only visible 
inside the function.

\headline{Conventions}

Variables of fundamental types are cheap to copy, it can be more 
efficient to pass them by value, therefore:
\begin{tightitemize}
  \item by default, pass variables of fundamental or enum type by value
  \item pass by reference if you want to modify an existing value
  \item pass a pointer if you want an optional "output" parameter
\end{tightitemize}

For variables of class type, the same rules apply, except that:
\begin{tightitemize}
  \item by default, pass by const reference to avoid copies
  \item pass by rvalue-reference if you want to transfer resource ownership
  \item pass by value if you want to work on a copy
\end{tightitemize}

\subsection*{Default arguments}

You may specify default arguments that are used in place of missing trailing arguments.

\bgroup
\def\arraystretch{1.5}
\begin{tabular}{ l } 
    \tsyntax{parameter-type} \tsyntax{parameter-name} = \tsyntax{default-value} \\ 
\end{tabular}
\egroup

Example:
\begin{codeblock}
void point(int x = 3, int y = 4);
point(1); // calls point(1, 4)
\end{codeblock}

\subsection*{Function overloading}

Two or more functions with the same name may be defined in the same scope, 
provided that their parameter list differ. Such functions are said to overload 
one another.

When calling an overloaded function, a process named overload resolution is performed 
to select the function that will be called. 
A simplified version of the process can be described as follow:
\begin{tightitemize}
  \item first, all functions that do not match the number of arguments provided, or for 
    which there is one or more arguments that cannot be converted to the parameter type are removed 
    from the set of candidate to form the set of viable functions;
  \item the viable functions are then ranked depending on the conversion they imply, 
    the fact that they are templates and other criterias in order to exhibit a best viable function
  \item if exactly one viable function is better than all the others, overload resolution succeeds 
    and this function is called; otherwise, compilation fails. 
\end{tightitemize}

\headline{deleted functions}

\bgroup
\def\arraystretch{1.5}
\begin{tabular}{ l } 
    \tsyntax{return-type} \tsyntax{name}(\tsyntax{parameter-list}) = delete; \tsincecxx{11} \\ 
\end{tabular}
\egroup

A function may be marked as deleted by replacing its body by \tcode{= delete;}.
Deleted functions participate in overload resolution, but if they are selected 
as the function to call, compilation fails.

\subsection*{Operator overloading}

\bgroup
\def\arraystretch{1.5}
\begin{tabular}{ l } 
    \tsyntax{return-type} operator \tsyntax{op}(\tsyntax{parameter-list}) \tsyntax{body} \\ 
\end{tabular}
\egroup

Much like regular functions, operators can be overloaded or marked as deleted. 
Here is a list of all possible values for \tsyntax{op}:
\vspace{-\parskip}
\begin{center}
    \tfcode{+} \tfcode{-} \tfcode{*} \tfcode{/} \tfcode{\%} \tfcode{\^} \tfcode{\&} 
    \tfcode{|} \tfcode{\~} \tfcode{!} \tfcode{=} \tfcode{<} \tfcode{>} \tfcode{+=} \tfcode{-=} 
    \tfcode{*=} \tfcode{/=} \tfcode{\%=} \tfcode{\^\/=} \tfcode{\&=} \tfcode{|=} \tfcode{<<}
    \tfcode{>>} \tfcode{>>=} \tfcode{<<=} \tfcode{==} \tfcode{!=} \tfcode{<=} \tfcode{>=} 
    \tfcode{\&\&} \tfcode{||} \tfcode{++} \tfcode{-\/-} \tfcode{,} \tfcode{->*} \tfcode{->} \tfcode{()}
    \tfcode{[]}\\ 
\end{center}

Although most operators can be overloaded, there is almost no practical use case for overloading 
say \tcode{operator,}. 
Programmers should be careful that overloaded operators keep a semantic close to the 
built-in operators.

Typical examples of operator overloading include \tcode{+}, \tcode{-}, \tcode{*} and
\tcode{/} for arithmetic types, \tcode{[]} for containers; \tcode{*} 
and \tcode{->} for smart-pointers; \tcode{++}, \tcode{-\/-} and \tcode{*}
for iterators; \tcode{<<} and \tcode{>>} for streams.


\subsection*{constexpr functions}

\bgroup
\def\arraystretch{1.5}
\begin{tabular}{ l } 
    constexpr \tsyntax{ret-type} \tsyntax{name}(\tsyntax{parameter-list}) \tsyntax{body} \tsincecxx{11} \\ 
\end{tabular}
\egroup

A function that fulfills the associated set of requirements may be marked as \tcode{constexpr}.
Such function can be evaluated at compile-time to produce constant-expressions.

Constexpr functions are most usefull for metaprogramming: "more metaprogramming, less 
template metaprogramming"!

See \href{https://en.cppreference.com/w/cpp/language/constexpr}{constexpr specifier} for the 
list of requirements. 

\section*{Additional Resources}

\href{https://en.cppreference.com/w/cpp/language/initialization}{Initialization}, \emph{en.cppreference.com}.

\href{https://en.cppreference.com/w/cpp/language/overload_resolution}{Overload resolution}, \emph{en.cppreference.com}.

\href{https://stackoverflow.com/questions/4421706/what-are-the-basic-rules-and-idioms-for-operator-overloading}{What are the basic rules and idioms for operator overloading?}, \emph{stackoverflow.com}.

\section*{Notes}

\begin{userendnotes}
  \hgrayrule
  \hgrayrule
  \hgrayrule
  \hgrayrule
  \hgrayrule
  \hgrayrule
  \hgrayrule
  \hgrayrule
  \hgrayrule
  \hgrayrule
  \hgrayrule
  \hgrayrule
  \hgrayrule
  \hgrayrule
  \hgrayrule
  \hgrayrule
  \hgrayrule
  \hgrayrule
  \hgrayrule
  \hgrayrule
  \hgrayrule
  \hgrayrule
  \hgrayrule
  \hgrayrule
\end{userendnotes}

\end{multicols*}

\end{document}