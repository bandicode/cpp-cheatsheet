\documentclass[10pt,landscape,a4paper]{article}
\usepackage[utf8]{inputenc}
\usepackage[english]{babel}

\usepackage[T1]{fontenc}
\usepackage{lmodern}
\usepackage{tikz}
\usetikzlibrary{shapes,positioning,arrows,fit,calc,graphs,graphs.standard}
%\usepackage[nosf]{kpfonts}
%\usepackage[t1]{sourcesanspro}
\usepackage{multicol}
\usepackage{wrapfig}
\usepackage[top=10mm,bottom=10mm,left=10mm,right=10mm]{geometry}
\usepackage[framemethod=tikz]{mdframed}
%\usepackage{microtype}
\usepackage{pdfpages}

\usepackage[pdftex,
            pdfauthor={bandicode},
            pdftitle={C++ Cheatsheet - Memory Management},
            pdfsubject={Memory Management},
            pdfkeywords={memory,smart pointers,allocators},
            pdfproducer={Latex},
            pdfcreator={pdflatex}]{hyperref}


\usepackage{array}
\usepackage{makecell}

\usepackage{enumitem}

\usepackage{colortbl} % provides arrayrulecolor
\usepackage{graphicx}
\usepackage{relsize}      % provide relative font size changes
\usepackage{xspace}


\renewcommand{\baselinestretch}{.8}
\pagestyle{empty}

\makeatletter 

\makeatother

\setlength{\parindent}{0pt}

\newenvironment{tightitemize}
{ \vspace{-\topsep}
  \vspace{4pt}
  \begin{itemize}[leftmargin=0pt, itemindent=12pt]
    \setlength{\itemsep}{4pt}
    \setlength{\parskip}{0pt}
    \setlength{\parsep}{0pt}      }
{ \end{itemize} \vspace{-\topsep} \vspace{4pt} } 

%%--------------------------------------------------
%% Sections & titles
\setlength{\parskip}{0.4\baselineskip}
\newcommand{\headline}[1]{\vspace{0.2\baselineskip}\textbf{#1}}

%%--------------------------------------------------
%% Horizontal gray rules

\newenvironment{userendnotes}[1]
  {
    \vspace{1ex}
    \setlength{\parskip}{1.2\baselineskip}
  }
  { 
    \setlength{\parskip}{0.4\baselineskip}
  }

\newcommand{\hgrayrule}{\par\noindent\textcolor{gray}{\rule{0.31\textwidth}{0.4pt}}}

%%--------------------------------------------------
%% array
\renewcommand\cellgape{\Gape[4pt]}
\newcolumntype{L}[1]{>{\raggedright\let\newline\\\arraybackslash\hspace{0pt}}m{#1}}
\newcolumntype{C}[1]{>{\centering\let\newline\\\arraybackslash\hspace{0pt}}m{#1}}
\newcolumntype{R}[1]{>{\raggedleft\let\newline\\\arraybackslash\hspace{0pt}}m{#1}}

%%--------------------------------------------------
%% Title
\newcommand{\CheatsheetTitle}[1]{
  \vbox { 
    \kern -6ex 
    \mbox{ 
      \includegraphics[scale=0.15]{common/cpp_logo} 
      \hspace{1em} 
      \raisebox{3ex}{
        \vbox{ 
          \hbox{\Huge \textsc{\Cpp Cheatsheet - #1} } 
          \kern 1ex
          \hbox{ \kern 0.1em Bibliography: \href{http://isocpp.org}{isocpp.org}, \href{http://cppreference.com}{cppreference.com}, \href{https://isocpp.github.io/CppCoreGuidelines/CppCoreGuidelines}{CppCoreGuidelines}}
        }
      }
    }
  }
}

\usepackage{titlesec}
\titlespacing\section{0pt}{\parskip}{-0.5\parskip}
\titlespacing\subsection{0pt}{\parskip}{-0.5\parskip}
\titlespacing\subsubsection{0pt}{\parskip}{-0.5\parskip}


%%--------------------------------------------------
%% fix interaction between hyperref and other
%% commands
\pdfstringdefDisableCommands{\def\smaller#1{#1}}
\pdfstringdefDisableCommands{\def\textbf#1{#1}}
\pdfstringdefDisableCommands{\def\raisebox#1{}}
\pdfstringdefDisableCommands{\def\hspace#1{}}


%%--------------------------------------------------
%% URL

\usepackage{hyperref}
\hypersetup{
    colorlinks=true,
    linkcolor=blue,
    filecolor=blue,      
    urlcolor=blue,
}
 
\urlstyle{same}

\newcommand{\hrefbf}[2]{\href{#1}{\textbf{#2}}}
\newcommand{\hrefcode}[2]{\href{#1}{\tcode{#2}}}
\newcommand{\hrefrequirements}[2]{\href{#1}{\textit{#2}}}
\newcommand{\hlink}[2]{\hyperlink{#1}{#2}}
\newcommand{\hlinkb}[2]{\hyperlink{#1}{\textbf{#2}}}



%%--------------------------------------------------
%% C++

\newcommand{\Rplus}{\protect\hspace{-.1em}\protect\raisebox{.35ex}{\smaller{\smaller\textbf{+}}}}
%\newcommand{\Rplus}{\protect\hspace{-.1em}\protect\raisebox{.25ex}{+}}
%\newcommand{\Rplus}{+}
\newcommand{\Cpp}{\mbox{C\Rplus\Rplus}\xspace}

%%--------------------------------------------------
%% Info

\definecolor{darkgreen}{RGB}{0,128,0}
\definecolor{darkorange}{RGB}{224, 96, 0}

\newcommand{\tmark}[1]{{\smaller\smaller \textcolor{darkgreen}{(#1)}}}
\newcommand{\tsincecxx}[1]{\tmark{since \Cpp\/#1}}
\newcommand{\tuntilcxx}[1]{\tmark{until \Cpp\/#1}}

\newcommand{\tmarkts}[1]{{\smaller\smaller \textcolor{darkorange}{(#1)}}}


% Inline code

\newcommand{\CodeStyle}{\ttfamily}
\newcommand{\CodeStylex}[1]{\texttt{#1}}
%\newcommand{\CodeStylex}[1]{\ttfamily{#1}}
%\newcommand{\CodeStylex}[1]{{\fontfamily{normalfont}\selectfont #1}}
\newcommand{\tcode}[1]{\CodeStylex{#1}}

% Inline code (framed)

\usepackage{xcolor}
\usepackage[most]{tcolorbox}

\definecolor{boxtfcode_border}{RGB}{214,214,214}
\definecolor{boxtfcode_bg}{RGB}{245,245,245}

\newtcbox{\boxtfcode}[1][boxtfcode_border]{on line,
	%size = tight,
    top=2pt,left=1pt,right=1pt,bottom=2pt,
    colback=boxtfcode_bg, colframe=boxtfcode_border, 
	boxsep=0pt, 
	boxrule=1pt, 
	arc=2pt
	}

\newcommand{\tfcode}[1]{\mbox{\boxtfcode{\tcode{#1}}}\xspace}

% Syntax

\newcommand{\tsyntax}[1]{\textcolor{gray}{\textit{#1}}}


%%--------------------------------------------------
%% Code listings

\usepackage{listings}
\tcbuselibrary{listings,skins}

\definecolor{main-color}{rgb}{0.6627, 0.7176, 0.7764}
\definecolor{back-color}{rgb}{0.1686, 0.1686, 0.1686}
%\definecolor{string-color}{rgb}{0.3333, 0.5254, 0.345}
\definecolor{string-color}{RGB}{51, 153, 0}
%\definecolor{key-color}{rgb}{0.8, 0.47, 0.196}
\definecolor{key-color}{RGB}{0,0,221}
\definecolor{preprocessor-color}{RGB}{51, 153, 0}
\definecolor{number-color}{RGB}{0, 0, 128}
\definecolor{comment-color}{RGB}{144,144,144}

\newcommand*{\FormatDigit}[1]{\textcolor{number-color}{#1}}

\lstset{language=C++,
        basicstyle=\ttfamily,
        keywordstyle=\color{key-color}\ttfamily,
        stringstyle=\color{string-color}\ttfamily,
        xleftmargin=0.5em,
        showstringspaces=false,
        commentstyle=\itshape\ttfamily,
        columns=flexible,
        keepspaces=true,
        texcl=true,
		tabsize=4,
		aboveskip=1pt,
		belowskip=0pt,
		literate={\ \ }{{\ }}1
		{0}{{\FormatDigit{0}}}{1}%
        {1}{{\FormatDigit{1}}}{1}%
        {2}{{\FormatDigit{2}}}{1}%
        {3}{{\FormatDigit{3}}}{1}%
        {4}{{\FormatDigit{4}}}{1}%
        {5}{{\FormatDigit{5}}}{1}%
        {6}{{\FormatDigit{6}}}{1}%
        {7}{{\FormatDigit{7}}}{1}%
        {8}{{\FormatDigit{8}}}{1}%
        {9}{{\FormatDigit{9}}}{1}%
        {.0}{{\FormatDigit{.0}}}{2}% Following is to ensure that only periods
        {.1}{{\FormatDigit{.1}}}{2}% followed by a digit are changed.
        {.2}{{\FormatDigit{.2}}}{2}%
        {.3}{{\FormatDigit{.3}}}{2}%
        {.4}{{\FormatDigit{.4}}}{2}%
        {.5}{{\FormatDigit{.5}}}{2}%
        {.6}{{\FormatDigit{.6}}}{2}%
        {.7}{{\FormatDigit{.7}}}{2}%
        {.8}{{\FormatDigit{.8}}}{2}%
        {.9}{{\FormatDigit{.9}}}{2}%
        %{,}{{\FormatDigit{,}}{1}% depends if you want the "," in color
		}

\newcommand{\CodeBlockSetup}{
 \lstset{escapechar=@}
}

%\lstnewenvironment{codeblock}{
%    \CodeBlockSetup
%}{}

\lstnewenvironment{hsynopsis}{
    \CodeBlockSetup
}{}

\newtcblisting{codeblock}[0]{
    top=0pt,left=2pt,right=2pt,bottom=0pt,
    colback=boxtfcode_bg, colframe=boxtfcode_border, 
	boxsep=0pt, 
	boxrule=1pt, 
	arc=2pt,
    listing only,
	listing options={language=C++,
        basicstyle=\small\CodeStyle,
        keywordstyle=\color{key-color}\small\CodeStyle,
		numberstyle=\color{number-color}\small\CodeStyle,
        stringstyle=\color{string-color}\small\CodeStyle,
        xleftmargin=0em,
        showstringspaces=false,
        %commentstyle=\itshape\rmfamily,
		commentstyle=\color{comment-color}\small\CodeStyle,
		directivestyle=\color{preprocessor-color}\small\CodeStyle,
        columns=flexible,
        keepspaces=true,
        texcl=true,
		tabsize=4,
        literate={\ \ }{{\ }}1
		{0}{{\FormatDigit{0}}}{1}%
        {1}{{\FormatDigit{1}}}{1}%
        {2}{{\FormatDigit{2}}}{1}%
        {3}{{\FormatDigit{3}}}{1}%
        {4}{{\FormatDigit{4}}}{1}%
        {5}{{\FormatDigit{5}}}{1}%
        {6}{{\FormatDigit{6}}}{1}%
        {7}{{\FormatDigit{7}}}{1}%
        {8}{{\FormatDigit{8}}}{1}%
        {9}{{\FormatDigit{9}}}{1}%
        {.0}{{\FormatDigit{.0}}}{2}% Following is to ensure that only periods
        {.1}{{\FormatDigit{.1}}}{2}% followed by a digit are changed.
        {.2}{{\FormatDigit{.2}}}{2}%
        {.3}{{\FormatDigit{.3}}}{2}%
        {.4}{{\FormatDigit{.4}}}{2}%
        {.5}{{\FormatDigit{.5}}}{2}%
        {.6}{{\FormatDigit{.6}}}{2}%
        {.7}{{\FormatDigit{.7}}}{2}%
        {.8}{{\FormatDigit{.8}}}{2}%
        {.9}{{\FormatDigit{.9}}}{2}%
        %{,}{{\FormatDigit{,}}{1}% depends if you want the "," in color
		},
}

% A code block in which single-quotes are digit separators
% rather than character literals.
\lstnewenvironment{codeblockdigitsep}{
 \CodeBlockSetup
 \lstset{deletestring=[b]{'}}
}{}

% Permit use of '@' inside codeblock blocks
\makeatletter
\newcommand{\atsign}{@}
\makeatother





\begin{document}

\CheatsheetTitle{Memory Management}

\small
\begin{multicols*}{3}

\section*{Manual memory management}

The keywords \tcode{\href{https://en.cppreference.com/w/cpp/language/new}{new}} and 
\tcode{\href{https://en.cppreference.com/w/cpp/language/delete}{delete}} can be used 
to create and destroy object on the heap.

\begin{codeblock}
Foo *foo = new Foo;
delete foo;
\end{codeblock}

Unlike stack-allocated arrays, arrays allocated on the heap can have sizes not known 
at compile time. Memory allocated with \tfcode{new[]} should be freed with \tfcode{delete[]}.

\begin{codeblock}
int *vec = new int[size];
delete[] vec;
\end{codeblock}

\headline{Initialization of dynamically allocated objects}

Objects allocated with \tfcode{new} can be initialized.

\begin{codeblock}
int *value = new int(42);
double *vec = new double[]{1.0, 2.0, 3.0};
\end{codeblock}

\subsection*{Placement new}

An object can be constructed in an already allocated storage using placement new.
This is done by providing the memory buffer to \tfcode{new}. 
Note that in such situation, the object should be destroyed by calling its destructor 
explicitly. 

\begin{codeblock}
char* ptr = new char[sizeof(T)];
T* tptr = new (ptr) T;
tptr->~T();
delete[] ptr;
\end{codeblock}

\section*{Automatic memory management}

Today's preferred way to deal with dynamically-allocated memory is 
through smart-pointers defined in header \tcode{<memory>}.

\subsection*{Smart pointers}

\bgroup
\def\arraystretch{1.5}
\def\cellwidth{6cm}
\begin{tabular}{ l L{\cellwidth} } 
 \arrayrulecolor{gray}\hline
   \hlinkb{unique-ptr-anchor}{\tcode{unique\_ptr}} & 
   smart pointer with unique object ownership semantics \tmark{class template} \\ 
 \hline
   \hlinkb{shared-ptr-anchor}{\tcode{shared\_ptr}} & 
   smart pointer with shared object ownership semantics \tmark{class template} \\ 
 \hline
   \hlinkb{weak-ptr-anchor}{\tcode{weak\_ptr}} & 
   weak reference to an object managed by \hlinkb{shared-ptr-anchor}{\tcode{shared\_ptr}} \tmark{class template} \\ 
\end{tabular}
\egroup

\subsubsection*{\hypertarget{unique-ptr-anchor}{\tcode{unique\_ptr}}}

The class template \hrefcode{https://en.cppreference.com/w/cpp/memory/unique_ptr}{std::unique\_ptr} 
owns an object that it destroys when the unique-pointer goes out of scope. 
This class satisfies the requirements \hrefrequirements{https://en.cppreference.com/w/cpp/named_req/MoveConstructible}{MoveConstructible} 
and \hrefrequirements{https://en.cppreference.com/w/cpp/named_req/MoveAssignable}{MoveAssignable} but it is neither 
copy constructible nor copyable. 

\bgroup
\def\arraystretch{0}
\def\cellwidth{7cm}
\begin{tabular}{ L{\cellwidth} l } 
 \arrayrulecolor{gray}\hline
   \begin{hsynopsis}
template<
    class T,
    class Deleter = std::default_delete<T>
> class unique_ptr;
   \end{hsynopsis}   & 
   (1) \\ 
\end{tabular}
\egroup

\vspace{-3ex}

The class overloads \tcode{operator*} and \tcode{operator->} so that it can 
be used as if it was a pointer to the underlying type.

\begin{codeblock}
{
  int* raw_ptr = new int(0);
  std::unique_ptr<int> ptr{raw_ptr};
  *ptr = 42;
  // the integer is destroyed by unique\_ptr at
  // the end of this block
}
\end{codeblock}

The class also provides a partial specialization for array types.

\begin{codeblock}
std::unique_ptr<int[]> ptr{new int[7]};
ptr[0] = 0;
\end{codeblock}

\headline{Essentials}

\bgroup
\def\arraystretch{1.5}
\def\cellwidth{6cm}
\begin{tabular}{ l L{\cellwidth} } 
 \arrayrulecolor{gray}\hline
    \hrefbf{https://en.cppreference.com/w/cpp/memory/unique_ptr/get}{\tcode{get}} & 
   returns a pointer to the managed object \tmark{public member function} \\ 
 \hline
   \hrefbf{https://en.cppreference.com/w/cpp/memory/unique_ptr/release}{\tcode{release}} & 
   returns a pointer to the managed object and releases the ownership  \tmark{public member function} \\ 
 \hline
   \hrefbf{https://en.cppreference.com/w/cpp/memory/unique_ptr/reset}{\tcode{reset}} & 
   replaces the managed object  \tmark{public member function} \\ 
\end{tabular}
\egroup

\subsubsection*{\hypertarget{shared-ptr-anchor}{\tcode{shared\_ptr}}}

The class template \hrefcode{https://en.cppreference.com/w/cpp/memory/shared_ptr}{std::shared\_ptr} 
provides a reference-counted pointer. 
When a shared pointer is copied, the reference counter is incremented. 
The managed object is destroyed when the last shared pointer referencing it is destroyed.

\bgroup
\def\arraystretch{0}
\def\cellwidth{8cm}
\begin{tabular}{ L{\cellwidth} l } 
 \arrayrulecolor{gray}\hline
   \begin{hsynopsis}
template<class T> class shared_ptr;
   \end{hsynopsis}   & \\ 
\end{tabular}
\egroup

\headline{Essentials}

\bgroup
\def\arraystretch{1.5}
\def\cellwidth{6cm}
\begin{tabular}{ l L{\cellwidth} } 
 \arrayrulecolor{gray}\hline
    \hrefbf{https://en.cppreference.com/w/cpp/memory/shared_ptr/get}{\tcode{get}} & 
   returns a pointer to the managed object \tmark{public member function} \\ 
 \hline
   \hrefbf{https://en.cppreference.com/w/cpp/memory/shared_ptr/reset}{\tcode{reset}} & 
   replaces the managed object \tmark{public member function} \\ 
 \hline
   \hrefbf{https://en.cppreference.com/w/cpp/memory/shared_ptr/use_count}{\tcode{use\_count}} & 
   returns the number of shared\_ptr objects referring to the same managed object  \tmark{public member function} \\ 
\end{tabular}
\egroup

Typical implementations contain a pointer to the managed object 
and a pointer to a \emph{control-block} that stores the number of 
shared pointers that own the object and the number of weak pointers 
that refer to the object.

\subsubsection*{\hypertarget{weak-ptr-anchor}{\tcode{weak\_ptr}}}

Weak pointers can be constructed from shared pointers to create a non-owning reference 
to a managed object.
Weak pointers can be used when temporary ownership is sufficient, or to 
break reference cycles.

\headline{Essentials}

\bgroup
\def\arraystretch{1.5}
\def\cellwidth{6cm}
\begin{tabular}{ l L{\cellwidth} } 
 \arrayrulecolor{gray}\hline
    \hrefbf{https://en.cppreference.com/w/cpp/memory/weak_ptr/expired}{\tcode{expired}} & 
   checks whether the referenced object was destroyed \tmark{public member function} \\ 
 \hline
   \hrefbf{https://en.cppreference.com/w/cpp/memory/weak_ptr/lock}{\tcode{lock}} & 
   creates a shared\_ptr that manages the referenced object \tmark{public member function} \\ 
\end{tabular}
\egroup

\headline{Use case: breaking reference cycles}

\begin{codeblock}
struct Node
{
  std::vector<std::shared_ptr<Node>> children;
  std::weak_ptr<Node> parent;

  Node(const std::shared_ptr<Node>& p = nullptr) 
    : parent(p) { }
  virtual ~Node() = default;
};
\end{codeblock}

\subsection*{Helper functions}

\bgroup
\def\arraystretch{1.5}
\def\cellwidth{6cm}
\begin{tabular}{ l L{\cellwidth} } 
 \arrayrulecolor{gray}\hline
   \hrefbf{https://en.cppreference.com/w/cpp/memory/shared_ptr/make_shared}{\tcode{make\_shared}} & 
   constructs an wrap an object in a shared pointer \tmark{function template} \\ 
 \hline
   \hrefbf{https://en.cppreference.com/w/cpp/memory/unique_ptr/make_unique}{\tcode{make\_unique}} & 
   constructs an wrap an object in a unique pointer \tmark{function template} \tsincecxx{14} \\ 
\end{tabular}
\egroup

When using \tcode{make\_shared}, only a single memory block large enough to store 
the managed object and the control block is allocated (the managed object is constructed in-place). 
Pros: you save a dynamic memory allocation compared to using \tcode{shared\_ptr} constructor; 
Cons: the allocated memory is released when the last weak pointer referencing the object 
is destroyed.

\subsection*{Helper class \tcode{enable\_shared\_from\_this}}

The \hrefcode{https://en.cppreference.com/w/cpp/memory/enable_shared_from_this/shared_from_this}{shared\_from\_this} 
function can be used to obtain a shared pointer from a raw pointer.
To enable this feature, your class must derive from
\hrefcode{https://en.cppreference.com/w/cpp/memory/enable_shared_from_this}{std::enable\_shared\_from\_this}.

\bgroup
\def\arraystretch{0}
\def\cellwidth{8cm}
\begin{tabular}{ L{\cellwidth} l } 
 \arrayrulecolor{gray}\hline
   \begin{hsynopsis}
template<class T> class enable_shared_from_this;
   \end{hsynopsis}   & \\ 
\end{tabular}
\egroup

Note that classes deriving from \tcode{enable\_shared\_from\_this} shouldn't be constructed on the stack.

\headline{Example:}

\begin{codeblock}
class Foo : 
  public std::enable_shared_from_this<Foo> 
{ 

};
/* ... */
Foo* foo = ...;
std::shared_ptr<Foo> ptr = foo->shared_from_this();
\end{codeblock}

\section*{Allocators}

Standard library containers use an \hrefrequirements{https://en.cppreference.com/w/cpp/named_req/Allocator}{Allocator} 
to allocate and release memory. 
The default allocator,
\hrefcode{https://en.cppreference.com/w/cpp/memory/allocator}{std::allocator}, is used 
if you don't specify any (and is sufficient most of the time).

\bgroup
\def\arraystretch{0}
\def\cellwidth{8cm}
\begin{tabular}{ L{\cellwidth} l } 
 \arrayrulecolor{gray}\hline
   \begin{hsynopsis}
template<class T>  struct allocator;
   \end{hsynopsis}   & \\ 
\end{tabular}
\egroup

\vspace{-3ex}

Custom allocators should at least provide the following:

\headline{Types}

\bgroup
\def\arraystretch{1.5}
\def\cellwidth{3cm}
\begin{tabular}{ L{\cellwidth} L{\cellwidth} } 
  \arrayrulecolor{gray}
  \hline
    \textbf{Type} & 
    \textbf{Definition} \\
  \hline
    \tcode{value\_type} & 
    \tfcode{T} \\
\end{tabular}
\egroup

\headline{Member functions}

\bgroup
\def\arraystretch{1.5}
\def\cellwidth{6cm}
\begin{tabular}{ l L{\cellwidth} } 
  \arrayrulecolor{gray}
  \hline    
    \hrefbf{https://en.cppreference.com/w/cpp/memory/allocator/allocate}{\tcode{allocate}} & 
    allocates uninitialized storage \tmark{public member function} \\ 
  \hline
    \hrefbf{https://en.cppreference.com/w/cpp/memory/allocator/deallocate}{\tcode{deallocate}} & 
    deallocates storage \tmark{public member function} \\ 
\end{tabular}
\egroup

\headline{Minimal allocator with debug output:}

\begin{codeblock}
template<class Tp>
struct Alloc {
  typedef Tp value_type;
  Alloc() = default;
  
  template<class T> Alloc(const Alloc<T>&) {}
  Tp* allocate(std::size_t n) 
  {
    n *= sizeof(Tp);
    std::cout << "allocating " << n << " bytes\n";
    return static_cast<Tp*>(::operator new(n));
  }
  
  void deallocate(Tp* p, std::size_t n) 
  {
    n *= sizeof(Tp);
    std::cout << "deallocating " << n << " bytes\n";
    ::operator delete(p);
  }
};
\end{codeblock}

\begin{codeblock}
template <class T, class U>
bool operator==(const Alloc<T>&, const Alloc<U>&) 
{ return true; }

template <class T, class U>
bool operator!=(const Alloc<T>&, const Alloc<U>&) 
{ return false; }
\end{codeblock}

You can specify various properties of your allocator by specializing the class template 
\hrefbf{https://en.cppreference.com/w/cpp/memory/allocator_traits}{\tcode{std::allocator\_traits}} 
for your allocator.

\section*{Overloading operator new \& delete}

You can track allocation and deallocation of user-defined types, or 
more generally customize the allocation process, by overloading
\hrefbf{https://en.cppreference.com/w/cpp/memory/new/operator_new}{\tcode{operator new}}.
This operator supports both global replacements and class-specific overloads.

These overloads are responsible for allocating and deallocating bytes of memory 
on which actual object are constructed.

\begin{codeblock}
struct X 
{
  static void* operator new(std::size_t sz) 
  {
    std::cout << "Allocating X (" << sz << " bytes)\n";
    return ::operator new(sz);
  }

  static void operator delete(void* ptr)
  {
    std::cout << "Deallocating X\n";
    ::operator delete(ptr);
  }
};
\end{codeblock}

\section*{Additional Resources}

\href{https://isocpp.org/wiki/faq/freestore-mgmt}{Memory Management}, \href{https://isocpp.org/faq}{C++ Super-FAQ}, \href{https://isocpp.org}{isocpp.org}.

\href{https://en.cppreference.com/w/cpp/memory/new}{Low level memory management}, \href{https://en.cppreference.com/}{en.cppreference.com}.

\vspace{40ex}

\section*{Notes}

\begin{userendnotes}
  \hgrayrule
  \hgrayrule
  \hgrayrule
  \hgrayrule
  \hgrayrule
  \hgrayrule
  \hgrayrule
  \hgrayrule
  \hgrayrule
  \hgrayrule
  \hgrayrule
  \hgrayrule
  \hgrayrule
  \hgrayrule
  \hgrayrule
  \hgrayrule
  \hgrayrule
  \hgrayrule
  \hgrayrule
  \hgrayrule
  \hgrayrule
  \hgrayrule
  \hgrayrule
  \hgrayrule
\end{userendnotes}

\end{multicols*}

\end{document}